\documentclass[a4paper,12pt]{article}

\input{header.tex}

\title{Отчёт по лабораторной работе \\ <<Система доменных имён>>}
\author{(Здесь писать Ф.~И.~О)}

\begin{document}

\maketitle

\tableofcontents

\section{Настройка системы DNS}

\subsection{Топология сети}

Топология сети и использыемые IP-адреса показаны на рис.~\ref{fig:network}.

\begin{figure}
\centering
\includegraphics[width=\textwidth]{includes/network_gv.pdf}
\caption{Топология сети}
\label{fig:network}
\end{figure}

\subsection{Структура службы доменных имён}

Структура авторитетных серверов доменных имён показана на рис.~\ref{fig:dns}.

\begin{figure}
\centering
\includegraphics[width=\textwidth]{includes/dns_gv.pdf}
\caption{Структура службы доменных имён}
\label{fig:dns}
\end{figure}

\subsection{Прочие настройки}

Кеширующие DNS-серверы
\begin{itemize}
\item ... ;
\item ...
\end{itemize}

Развёрнутые SMTP-серверы и используемые ими кеширующие DNS-серверы.
\begin{itemize}
\item ... использует сервер на ... ;
\item ...
\end{itemize}


\section{Проверка настройки службы доменных имён}

\subsection{Проверка настройки записи типа ... для домена ...}

(по цепочке опрашиваем с корневого сервера ручками)

\begin{verbatim}
первый вывод dig
\end{verbatim}

\begin{verbatim}
второй вывод dig
\end{verbatim}

...

Итоговая проверка: опрашиваем кеширующий DNS-сервер.

\begin{verbatim}
Здесь вывод dig при опросе кеширующего сервера.
\end{verbatim}

\begin{verbatim}
Здесь вывод ping для искомого имени.
\end{verbatim}

\subsection{Проверка настройки записи типа ... для домена ...}

Повторяем далее.

\section{Проверка работы почтовой системы}

\subsection{Проверка MX-записи для домена ...}

С узла ... отправили письмо на локальный SMTP-сервер для адресата с адресом .....

\begin{verbatim}
Сюда нужно поместить лог работы с SMTP-сервером.
\end{verbatim}

На машине с доменным именем ... появилось доставленное письмо.
\begin{verbatim}
Результат cat /var/mail/...
\end{verbatim}

Таким образом, доменная запись типа MX для домена ... настроена верно.

\subsection{Проверка MX-записи для домена ...}

(повторить для всех SMTP-серверов)

\end{document}
