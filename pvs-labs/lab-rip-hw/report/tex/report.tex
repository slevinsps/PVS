\documentclass[a4paper,12pt]{article}

\input{header.tex}

\title{Отчёт по лабораторной работе \\ <<Динамическая IP-маршрутизация>>}
\author{(Здесь писать Ф.~И.~О)}

\begin{document}

\maketitle

\tableofcontents

\section{Настройка сети}

\subsection{Топология сети}

Топология сети и используемые IP-адреса показаны на рисунке~\ref{fig:network}.

\begin{figure}
\centering
\includegraphics[width=0.8\textwidth]{includes/network_gv.pdf}
\caption{Топология сети}
\label{fig:network}
\end{figure}
\newpage

Перечень узлов, на которых используется динамическая IP-маршрутизация: r1, r2, r3, r4, r5.


\subsection{Назначение IP-адресов}

Ниже приведён файл сетевой настройки  маршрутизатора r1.

\begin{Verbatim}
auto lo
iface lo inet loopback

auto eth0
iface eth0 inet static
address 10.10.2.2
netmask 255.255.255.0

auto eth1
iface eth1 inet static
address 10.10.4.1
netmask 255.255.255.0

auto eth2
iface eth2 inet static
address 10.10.1.2
netmask 255.255.255.0

\end{Verbatim}

Ниже приведён файл сетевой настройки  маршрутизатора r2.

\begin{Verbatim}
auto lo
iface lo inet loopback

auto eth0
iface eth0 inet static
address 10.10.4.2
netmask 255.255.255.0

auto eth1
iface eth1 inet static
address 10.10.5.3
netmask 255.255.255.0
\end{Verbatim}

Ниже приведён файл сетевой настройки  маршрутизатора r3.

\begin{Verbatim}
auto lo
iface lo inet loopback

auto eth0
iface eth0 inet static
address 10.10.3.2
netmask 255.255.255.0

auto eth1
iface eth1 inet static
address 10.10.5.2
netmask 255.255.255.0


\end{Verbatim}

Ниже приведён файл сетевой настройки  маршрутизатора r4.

\begin{Verbatim}
auto lo
iface lo inet loopback

auto eth0
iface eth0 inet static
address 10.10.2.1
netmask 255.255.255.0

auto eth1
iface eth1 inet static
address 10.10.3.1
netmask 255.255.255.0

auto eth2
iface eth2 inet static
address 10.10.1.1
netmask 255.255.255.0

\end{Verbatim}

Ниже приведён файл сетевой настройки  маршрутизатора r5.

\begin{Verbatim}
auto lo
iface lo inet loopback

auto eth0
iface eth0 inet static
address 10.10.5.1
netmask 255.255.255.0

auto eth1
iface eth1 inet static
address 10.10.6.2
netmask 255.255.255.0

\end{Verbatim}



Ниже приведён файл сетевой настройки рабочей станции ws2.

\begin{Verbatim}
auto lo
iface lo inet loopback

auto eth0
iface eth0 inet static
address 10.10.6.1
netmask 255.255.255.0
gateway 10.10.6.2

\end{Verbatim}

Ниже приведён файл сетевой настройки рабочей станции wsp1.

\begin{Verbatim}
auto lo
iface lo inet loopback

auto eth0
iface eth0 inet static
address 10.10.1.1
netmask 255.255.255.0

\end{Verbatim}


\subsection{Настройка протокола RIP}

Ниже приведен файл \Code{/etc/quagga/ripd.conf} маршрутизатора r1.

\begin{Verbatim}
router rip

network eth0
network eth1
network eth2

timers basic 10 60 120

redistribute kernel
redistribute connected

log file /var/log/quagga/ripd.log

\end{Verbatim}

Ниже приведен файл \Code{/etc/quagga/ripd.conf} маршрутизатора r2.

\begin{Verbatim}
router rip

network eth0
network eth1

timers basic 10 60 120

redistribute kernel
! redistribute connected

log file /var/log/quagga/ripd.log

\end{Verbatim}

Ниже приведен файл \Code{/etc/quagga/ripd.conf} маршрутизатора r3.

\begin{Verbatim}
router rip

network eth0
network eth1

timers basic 10 60 120

redistribute kernel
redistribute connected

log file /var/log/quagga/ripd.log

\end{Verbatim}

Ниже приведен файл \Code{/etc/quagga/ripd.conf} маршрутизатора r4.

\begin{Verbatim}
router rip

network eth0
network eth1
network eth2

timers basic 10 60 120


redistribute kernel
redistribute connected

log file /var/log/quagga/ripd.log

\end{Verbatim}

Ниже приведен файл \Code{/etc/quagga/ripd.conf} маршрутизатора r5.

\begin{Verbatim}
router rip

network eth0

timers basic 10 60 120

redistribute kernel
redistribute connected

log file /var/log/quagga/ripd.log

\end{Verbatim}


Ниже приведен файл \Code{/etc/quagga/ripd.conf} рабочий станции, связанной с несколькими маршрутизаторами wsp1.

\begin{Verbatim}
router rip

network eth0

timers basic 10 60 120

redistribute kernel
redistribute connected

log file /var/log/quagga/ripd.log

\end{Verbatim}


\section{Проверка настройки протокола RIP}

Вывод \textbf{traceroute} от узла ws2 до wsp1 при нормальной работе сети.

\begin{Verbatim}
traceroute to 10.10.1.1 (10.10.1.1), 64 hops max, 40 byte packets
 1  10.10.6.2 (10.10.6.2)  7 ms  1 ms  1 ms
 2  10.10.5.2 (10.10.5.2)  11 ms  2 ms  1 ms
 3  10.10.3.1 (10.10.3.1)  7 ms  1 ms  1 ms
 4  10.10.1.1 (10.10.1.1)  13 ms  2 ms  2 ms
\end{Verbatim}

Вывод \textbf{traceroute} от узла r2 до внешнего IP 87.250.250.242.

\begin{Verbatim}
traceroute to 87.250.250.242 (87.250.250.242), 64 hops max, 40 byte packets
 1  172.16.0.1 (172.16.0.1)  0 ms  0 ms  0 ms
 2  172.19.233.1 (172.19.233.1)  2 ms  2 ms  2 ms
 3  172.16.3.12 (172.16.3.12)  2 ms  2 ms  2 ms
 4  188.72.78.17 (188.72.78.17)  2 ms  1 ms  2 ms
 5  83.220.58.221 (83.220.58.221)  2 ms  3 ms  3 ms
 6  94.124.192.3 (94.124.192.3)  6 ms  5 ms  4 ms
 7  * * *
 8  87.250.250.242 (87.250.250.242)  7 ms (TOS=96!) *  6 ms
 
\end{Verbatim}

Вывод сообщения RIP на r3.

\begin{Verbatim}
IP (tos 0x0, ttl 1, id 0, offset 0, flags [DF], proto UDP (17), length 112) 10.10.3.1.520 > 224.0.0.9.520: 
	RIPv2, Response, length: 84, routes: 4
	  AFI: IPv4:         0.0.0.0/0 , tag 0x0000, metric: 3, next-hop: self
	  AFI: IPv4:       10.10.1.0/24, tag 0x0000, metric: 1, next-hop: self
	  AFI: IPv4:       10.10.2.0/24, tag 0x0000, metric: 1, next-hop: self
	  AFI: IPv4:       10.10.4.0/24, tag 0x0000, metric: 2, next-hop: self
IP (tos 0x0, ttl 1, id 0, offset 0, flags [DF], proto UDP (17), length 112) 10.10.3.2.520 > 224.0.0.9.520: 
	RIPv2, Response, length: 84, routes: 4
	  AFI: IPv4:         0.0.0.0/0 , tag 0x0000, metric: 2, next-hop: self
	  AFI: IPv4:       10.10.4.0/24, tag 0x0000, metric: 2, next-hop: self
	  AFI: IPv4:       10.10.5.0/24, tag 0x0000, metric: 1, next-hop: self
	  AFI: IPv4:       10.10.6.0/24, tag 0x0000, metric: 2, next-hop: self
\end{Verbatim}

Вывод таблицы RIP на r3.

\begin{Verbatim}
Codes: R - RIP, C - connected, S - Static, O - OSPF, B - BGP
Sub-codes:
      (n) - normal, (s) - static, (d) - default, (r) - redistribute,
      (i) - interface

     Network            Next Hop         Metric From            Tag Time
R(n) 0.0.0.0/0          10.10.5.3             2 10.10.5.3         0 00:13
R(n) 10.10.1.0/24       10.10.3.1             2 10.10.3.1         0 00:13
R(n) 10.10.2.0/24       10.10.3.1             2 10.10.3.1         0 00:13
C(i) 10.10.3.0/24       0.0.0.0               1 self              0
R(n) 10.10.4.0/24       10.10.5.3             2 10.10.5.3         0 00:13
C(i) 10.10.5.0/24       0.0.0.0               1 self              0
R(n) 10.10.6.0/24       10.10.5.1             2 10.10.5.1         0 00:12

\end{Verbatim}

Вывод таблицы маршрутизации на r3.

\begin{Verbatim}
10.10.6.0/24 via 10.10.5.1 dev eth1  proto zebra  metric 2 
10.10.4.0/24 via 10.10.5.3 dev eth1  proto zebra  metric 2 
10.10.5.0/24 dev eth1  proto kernel  scope link  src 10.10.5.2 
10.10.2.0/24 via 10.10.3.1 dev eth0  proto zebra  metric 2 
10.10.3.0/24 dev eth0  proto kernel  scope link  src 10.10.3.2 
10.10.1.0/24 via 10.10.3.1 dev eth0  proto zebra  metric 2 
default via 10.10.5.3 dev eth1  proto zebra  metric 2 
\end{Verbatim}

\section{Расщепленный горизонт и испорченные обратные обновления}

Включенный расщепленный горизонт:
\begin{Verbatim}
IP (tos 0x0, ttl 1, id 0, offset 0, flags [DF], proto UDP (17), length 112) 10.10.3.2.520 > 224.0.0.9.520: 
	RIPv2, Response, length: 84, routes: 4
	  AFI: IPv4:         0.0.0.0/0 , tag 0x0000, metric: 2, next-hop: self
	  AFI: IPv4:       10.10.4.0/24, tag 0x0000, metric: 2, next-hop: self
	  AFI: IPv4:       10.10.5.0/24, tag 0x0000, metric: 1, next-hop: self
	  AFI: IPv4:       10.10.6.0/24, tag 0x0000, metric: 2, next-hop: self
\end{Verbatim}


Включены испорченные обратные направления:
\begin{Verbatim}

IP (tos 0x0, ttl 1, id 0, offset 0, flags [DF], proto UDP (17), length 172) 10.10.3.2.520 > 224.0.0.9.520: 
	RIPv2, Response, length: 144, routes: 7
	  AFI: IPv4:         0.0.0.0/0 , tag 0x0000, metric: 2, next-hop: self
	  AFI: IPv4:       10.10.1.0/24, tag 0x0000, metric: 16, next-hop: 10.10.3.1
	  AFI: IPv4:       10.10.2.0/24, tag 0x0000, metric: 16, next-hop: 10.10.3.1
	  AFI: IPv4:       10.10.3.0/24, tag 0x0000, metric: 16, next-hop: self
	  AFI: IPv4:       10.10.4.0/24, tag 0x0000, metric: 2, next-hop: self
	  AFI: IPv4:       10.10.5.0/24, tag 0x0000, metric: 1, next-hop: self
	  AFI: IPv4:       10.10.6.0/24, tag 0x0000, metric: 2, next-hop: self
	  
\end{Verbatim}

С отключенным расщепленным горизонтом:
\begin{Verbatim}
IP (tos 0x0, ttl 1, id 0, offset 0, flags [DF], proto UDP (17), length 172) 10.10.3.2.520 > 224.0.0.9.520: 
	RIPv2, Response, length: 144, routes: 7
	  AFI: IPv4:         0.0.0.0/0 , tag 0x0000, metric: 2, next-hop: self
	  AFI: IPv4:       10.10.1.0/24, tag 0x0000, metric: 2, next-hop: 10.10.3.1
	  AFI: IPv4:       10.10.2.0/24, tag 0x0000, metric: 2, next-hop: 10.10.3.1
	  AFI: IPv4:       10.10.3.0/24, tag 0x0000, metric: 1, next-hop: self
	  AFI: IPv4:       10.10.4.0/24, tag 0x0000, metric: 2, next-hop: self
	  AFI: IPv4:       10.10.5.0/24, tag 0x0000, metric: 1, next-hop: self
	  AFI: IPv4:       10.10.6.0/24, tag 0x0000, metric: 2, next-hop: self	
\end{Verbatim}



\section{Имитация устранимой поломки в сети}

Был выключен маршрутизатор r4.

Вывод таблицы RIP непосредственно перед истечением таймера устаревания на маршрутизаторе r3.

\begin{Verbatim}
Codes: R - RIP, C - connected, S - Static, O - OSPF, B - BGP
Sub-codes:
      (n) - normal, (s) - static, (d) - default, (r) - redistribute,
      (i) - interface

     Network            Next Hop         Metric From            Tag Time
R(n) 0.0.0.0/0          10.10.5.3             2 10.10.5.3         0 00:12
R(n) 10.10.1.0/24       10.10.3.1             2 10.10.3.1         0 00:00
R(n) 10.10.2.0/24       10.10.3.1             2 10.10.3.1         0 00:00
C(i) 10.10.3.0/24       0.0.0.0               1 self              0
R(n) 10.10.4.0/24       10.10.5.3             2 10.10.5.3         0 00:12
C(i) 10.10.5.0/24       0.0.0.0               1 self              0
R(n) 10.10.6.0/24       10.10.5.1             2 10.10.5.1         0 00:14
\end{Verbatim}

Перестроенная таблица на этом же маршрутизаторе

\begin{Verbatim}
Codes: R - RIP, C - connected, S - Static, O - OSPF, B - BGP
Sub-codes:
      (n) - normal, (s) - static, (d) - default, (r) - redistribute,
      (i) - interface

     Network            Next Hop         Metric From            Tag Time
R(n) 0.0.0.0/0          10.10.5.3             2 10.10.5.3         0 00:19
R(n) 10.10.1.0/24       10.10.5.3             3 10.10.5.3         0 00:19
R(n) 10.10.2.0/24       10.10.5.3             3 10.10.5.3         0 00:19
C(i) 10.10.3.0/24       0.0.0.0               1 self              0
R(n) 10.10.4.0/24       10.10.5.3             2 10.10.5.3         0 00:19
C(i) 10.10.5.0/24       0.0.0.0               1 self              0
R(n) 10.10.6.0/24       10.10.5.1             2 10.10.5.1         0 00:11
\end{Verbatim}


Вывод \textbf{traceroute} от узла ws2 до wsp1 после того, как служба RIP перестроила таблицы маршрутизации.

\begin{Verbatim}
traceroute to 10.10.1.1 (10.10.1.1), 64 hops max, 40 byte packets
 1  10.10.6.2 (10.10.6.2)  11 ms  1 ms  1 ms
 2  10.10.5.3 (10.10.5.3)  12 ms  1 ms  1 ms
 3  10.10.4.1 (10.10.4.1)  13 ms  2 ms  1 ms
 4  10.10.1.1 (10.10.1.1)  13 ms  2 ms  1 ms
\end{Verbatim}

\section{Имитация неустранимой поломки в сети}

Был отключен маршрутизатор r1.
\begin{Verbatim}

     Network            Next Hop         Metric From            Tag Time
R(n) 0.0.0.0/0          10.10.5.3             2 10.10.5.3         0 00:12
R(n) 10.10.1.0/24       10.10.5.3            16 10.10.5.3         0 00:02
R(n) 10.10.2.0/24       10.10.5.3            16 10.10.5.3         0 00:00
C(i) 10.10.3.0/24       0.0.0.0               1 self              0
R(n) 10.10.4.0/24       10.10.5.3             2 10.10.5.3         0 00:12
C(i) 10.10.5.0/24       0.0.0.0               1 self              0
R(n) 10.10.6.0/24       10.10.5.1             2 10.10.5.1         0 00:11
\end{Verbatim}

\begin{Verbatim}
     Network            Next Hop         Metric From            Tag Time
R(n) 0.0.0.0/0          10.10.5.3             2 10.10.5.3         0 00:19
C(i) 10.10.3.0/24       0.0.0.0               1 self              0
R(n) 10.10.4.0/24       10.10.5.3             2 10.10.5.3         0 00:19
C(i) 10.10.5.0/24       0.0.0.0               1 self              0
R(n) 10.10.6.0/24       10.10.5.1             2 10.10.5.1         0 00:18
\end{Verbatim}


\end{document}
