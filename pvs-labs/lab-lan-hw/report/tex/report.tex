
\documentclass[a4paper,12pt]{article}

\usepackage[utf8x]{inputenc}
\usepackage[T2A]{fontenc}
\usepackage[english, russian]{babel}

% Опционно, требует  apt-get install scalable-cyrfonts.*
% и удаления одной строчки в cyrtimes.sty
% Сточку не удалять!
% \usepackage{cyrtimes}

% Картнки и tikz
\usepackage{graphicx}
\usepackage{tikz}
\usetikzlibrary{snakes,arrows,shapes}


% Некоторая русификация.
\usepackage{misccorr}
\usepackage{indentfirst}
\renewcommand{\labelitemi}{\normalfont\bfseries{--}}

% Увы, поля придётся уменьшить из-за листингов.
\topmargin -1cm
\oddsidemargin -0.5cm
\evensidemargin -0.5cm
\textwidth 17cm
\textheight 24cm

\sloppy

% Оглавление в PDF
\usepackage[
bookmarks=true,
colorlinks=true, linkcolor=black, anchorcolor=black, citecolor=black, menucolor=black,filecolor=black, urlcolor=black,
unicode=true
]{hyperref}

% Для исходного кода в тексте
\newcommand{\Code}[1]{\texttt{#1}}


\title{Отчёт по лабораторной работе \\ <<Локальные сети>>}
\author{Здесь Ф.~И.~О}

\begin{document}

\maketitle

\tableofcontents

% Текст отчёта должен быть читаемым!!! Написанное здесь является рыбой.

\section{Получение адреса по DHCP}

Дампим командой tcpdump -tenv -s 1000 -i eth0 udp на R2, получение случайного адреса

\begin{Verbatim}
10:10:10:10:10:ee > ff:ff:ff:ff:ff:ff, ethertype IPv4 (0x0800), length 342: (tos 0x10, ttl 128, id 0, offset 0, flags [none], proto UDP (17), length 328) 0.0.0.0.68 > 255.255.255.255.67: BOOTP/DHCP, Request from 10:10:10:10:10:ee, length 300, xid 0x6ff4e753, Flags [none]
	  Client-Ethernet-Address 10:10:10:10:10:ee
	  Vendor-rfc1048 Extensions
	    Magic Cookie 0x63825363
	    DHCP-Message Option 53, length 1: Discover
	    Parameter-Request Option 55, length 12: 
	      Subnet-Mask, BR, Time-Zone, Default-Gateway
	      Domain-Name, Domain-Name-Server, Option 119, Hostname
	      Netbios-Name-Server, Netbios-Scope, MTU, Classless-Static-Route
3a:40:ee:31:9e:cd > 10:10:10:10:10:ee, ethertype IPv4 (0x0800), length 342: (tos 0x10, ttl 128, id 0, offset 0, flags [none], proto UDP (17), length 328) 10.20.0.1.67 > 10.20.0.2.68: BOOTP/DHCP, Reply, length 300, xid 0x6ff4e753, Flags [none]
	  Your-IP 10.20.0.2
	  Client-Ethernet-Address 10:10:10:10:10:ee
	  Vendor-rfc1048 Extensions
	    Magic Cookie 0x63825363
	    DHCP-Message Option 53, length 1: Offer
	    Server-ID Option 54, length 4: 10.20.0.1
	    Lease-Time Option 51, length 4: 43200
	    Subnet-Mask Option 1, length 4: 255.255.0.0
	    Default-Gateway Option 3, length 4: 10.20.0.1
	    Domain-Name-Server Option 6, length 4: 10.20.0.1
10:10:10:10:10:ee > ff:ff:ff:ff:ff:ff, ethertype IPv4 (0x0800), length 342: (tos 0x10, ttl 128, id 0, offset 0, flags [none], proto UDP (17), length 328) 0.0.0.0.68 > 255.255.255.255.67: BOOTP/DHCP, Request from 10:10:10:10:10:ee, length 300, xid 0x6ff4e753, Flags [none]
	  Client-Ethernet-Address 10:10:10:10:10:ee
	  Vendor-rfc1048 Extensions
	    Magic Cookie 0x63825363
	    DHCP-Message Option 53, length 1: Request
	    Server-ID Option 54, length 4: 10.20.0.1
	    Requested-IP Option 50, length 4: 10.20.0.2
	    Parameter-Request Option 55, length 12: 
	      Subnet-Mask, BR, Time-Zone, Default-Gateway
	      Domain-Name, Domain-Name-Server, Option 119, Hostname
	      Netbios-Name-Server, Netbios-Scope, MTU, Classless-Static-Route
3a:40:ee:31:9e:cd > 10:10:10:10:10:ee, ethertype IPv4 (0x0800), length 342: (tos 0x10, ttl 128, id 0, offset 0, flags [none], proto UDP (17), length 328) 10.20.0.1.67 > 10.20.0.2.68: BOOTP/DHCP, Reply, length 300, xid 0x6ff4e753, Flags [none]
	  Your-IP 10.20.0.2
	  Client-Ethernet-Address 10:10:10:10:10:ee
	  Vendor-rfc1048 Extensions
	    Magic Cookie 0x63825363
	    DHCP-Message Option 53, length 1: ACK
	    Server-ID Option 54, length 4: 10.20.0.1
	    Lease-Time Option 51, length 4: 43200
	    Subnet-Mask Option 1, length 4: 255.255.0.0
	    Default-Gateway Option 3, length 4: 10.20.0.1
	    Domain-Name-Server Option 6, length 4: 10.20.0.1

\end{Verbatim}

Дампим командой tcpdump -tenv -s 1000 -i eth0 udp на R1, получение фиксированного адреса

\begin{Verbatim}
10:10:10:10:20:aa > ff:ff:ff:ff:ff:ff, ethertype IPv4 (0x0800), length 342: (tos 0x10, ttl 128, id 0, offset 0, flags [none], proto UDP (17), length 328) 0.0.0.0.68 > 255.255.255.255.67: BOOTP/DHCP, Request from 10:10:10:10:20:aa, length 300, xid 0x9e66251b, Flags [none]
	  Client-Ethernet-Address 10:10:10:10:20:aa
	  Vendor-rfc1048 Extensions
	    Magic Cookie 0x63825363
	    DHCP-Message Option 53, length 1: Discover
	    Requested-IP Option 50, length 4: 10.10.4.10
	    Parameter-Request Option 55, length 12: 
	      Subnet-Mask, BR, Time-Zone, Default-Gateway
	      Domain-Name, Domain-Name-Server, Option 119, Hostname
	      Netbios-Name-Server, Netbios-Scope, MTU, Classless-Static-Route
0e:ab:f8:0c:10:4b > 10:10:10:10:20:aa, ethertype IPv4 (0x0800), length 342: (tos 0x10, ttl 128, id 0, offset 0, flags [none], proto UDP (17), length 328) 10.10.0.1.67 > 10.10.4.10.68: BOOTP/DHCP, Reply, length 300, xid 0x9e66251b, Flags [none]
	  Your-IP 10.10.4.10
	  Client-Ethernet-Address 10:10:10:10:20:aa
	  Vendor-rfc1048 Extensions
	    Magic Cookie 0x63825363
	    DHCP-Message Option 53, length 1: Offer
	    Server-ID Option 54, length 4: 10.10.0.1
	    Lease-Time Option 51, length 4: 43200
	    Subnet-Mask Option 1, length 4: 255.255.0.0
	    Default-Gateway Option 3, length 4: 10.10.0.1
	    Domain-Name-Server Option 6, length 4: 10.10.0.1
10:10:10:10:20:aa > ff:ff:ff:ff:ff:ff, ethertype IPv4 (0x0800), length 342: (tos 0x10, ttl 128, id 0, offset 0, flags [none], proto UDP (17), length 328) 0.0.0.0.68 > 255.255.255.255.67: BOOTP/DHCP, Request from 10:10:10:10:20:aa, length 300, xid 0x9e66251b, Flags [none]
	  Client-Ethernet-Address 10:10:10:10:20:aa
	  Vendor-rfc1048 Extensions
	    Magic Cookie 0x63825363
	    DHCP-Message Option 53, length 1: Request
	    Server-ID Option 54, length 4: 10.10.0.1
	    Requested-IP Option 50, length 4: 10.10.4.10
	    Parameter-Request Option 55, length 12: 
	      Subnet-Mask, BR, Time-Zone, Default-Gateway
	      Domain-Name, Domain-Name-Server, Option 119, Hostname
	      Netbios-Name-Server, Netbios-Scope, MTU, Classless-Static-Route
0e:ab:f8:0c:10:4b > 10:10:10:10:20:aa, ethertype IPv4 (0x0800), length 342: (tos 0x10, ttl 128, id 0, offset 0, flags [none], proto UDP (17), length 328) 10.10.0.1.67 > 10.10.4.10.68: BOOTP/DHCP, Reply, length 300, xid 0x9e66251b, Flags [none]
	  Your-IP 10.10.4.10
	  Client-Ethernet-Address 10:10:10:10:20:aa
	  Vendor-rfc1048 Extensions
	    Magic Cookie 0x63825363
	    DHCP-Message Option 53, length 1: ACK
	    Server-ID Option 54, length 4: 10.10.0.1
	    Lease-Time Option 51, length 4: 43200
	    Subnet-Mask Option 1, length 4: 255.255.0.0
	    Default-Gateway Option 3, length 4: 10.10.0.1
	    Domain-Name-Server Option 6, length 4: 10.10.0.1
\end{Verbatim}


\section{Использование VPN}
ip r на маршрутизаторе R1 после VPN и работы RIP
\begin{Verbatim}
10.100.100.2 dev tun0  proto kernel  scope link  src 10.100.100.1 
10.20.0.0/16 via 10.100.100.2 dev tun0  proto zebra  metric 2 
10.10.0.0/16 dev eth0  proto kernel  scope link  src 10.10.0.1 
172.16.0.0/16 dev eth1  proto kernel  scope link  src 172.16.1.3 
default via 172.16.1.2 dev eth1 
\end{Verbatim}

ip -4 a на маршрутизаторе R1
\begin{Verbatim}
1: lo: <LOOPBACK,UP,LOWER_UP> mtu 16436 qdisc noqueue 
    inet 127.0.0.1/8 scope host lo
3: eth1: <BROADCAST,MULTICAST,UP,LOWER_UP> mtu 1500 qdisc pfifo_fast qlen 1000
    inet 172.16.1.3/16 brd 172.16.255.255 scope global eth1
4: eth0: <BROADCAST,MULTICAST,UP,LOWER_UP> mtu 1500 qdisc pfifo_fast qlen 1000
    inet 10.10.0.1/16 brd 10.10.255.255 scope global eth0
5: tun0: <POINTOPOINT,MULTICAST,NOARP,UP,LOWER_UP> mtu 1500 qdisc pfifo_fast qlen 100
    inet 10.100.100.1 peer 10.100.100.2/32 scope global tun0
\end{Verbatim}

просшулка сообщений RIP на tun0
tcpdump -tvn -i tun0 -s 1518 udp
\begin{Verbatim}
IP (tos 0x0, ttl 1, id 0, offset 0, flags [DF], proto UDP (17), length 52) 10.100.100.2.520 > 224.0.0.9.520: 
	RIPv2, Response, length: 24, routes: 1
	  AFI: IPv4:       10.20.0.0/16, tag 0x0000, metric: 1, next-hop: self
IP (tos 0x0, ttl 1, id 0, offset 0, flags [DF], proto UDP (17), length 52) 10.100.100.1.520 > 224.0.0.9.520: 
	RIPv2, Response, length: 24, routes: 1
	  AFI: IPv4:       10.10.0.0/16, tag 0x0000, metric: 1, next-hop: self
\end{Verbatim}

Проверка работы VPN

Трейс с ws21 до s11
\begin{Verbatim}
traceroute to 10.10.4.10 (10.10.4.10), 64 hops max, 40 byte packets
 1  10.20.0.1 (10.20.0.1)  7 ms  1 ms  11 ms
 2  10.100.100.1 (10.100.100.1)  3 ms  3 ms  3 ms
 3  10.10.4.10 (10.10.4.10)  14 ms  4 ms  3 ms
\end{Verbatim}

\section{Правила фильтации пакетов и трансляции пдресов}

\begin{Verbatim}
#!/bin/sh
LAN=eth0
INET=eth1
VPN=tun0
# Удаление всех правил в таблице "filter" (по-умолчанию).
iptables -F
# Удаление правил в таблице "nat" (её надо указать явно).
iptables -F -t nat
# По-умолчанию все маршрутизируемые пакеты выбрасываются.
iptables --policy FORWARD DROP
# ICMP разрешим
iptables -A FORWARD -p icmp -j ACCEPT
# Разрешаем любую маршрутизацию для интерфейса VPN
iptables -A FORWARD -i $VPN -j ACCEPT
iptables -A FORWARD -o $VPN -j ACCEPT
# Включение SNAT для маршрутизируемых пакетов, выходящих
# через eth1. Это правило выполняется после самой маршрутизации
# (POSTROUTING) и помещается в таблицу правил "nat".
iptables -t nat -A POSTROUTING -o $INET -j MASQUERADE
# Разрешение пакетов-ответов (они отслеживаются как
# -- state ESTABLISHED)
iptables -A FORWARD -m state --state ESTABLISHED -i $INET -j ACCEPT

iptables -A FORWARD -s 10.10.4.10 -p tcp --dport 80 -j ACCEPT
iptables -A FORWARD -s 10.10.4.20 -j ACCEPT


iptables -t nat -A PREROUTING -p tcp --dport 80 -j DNAT --to 10.10.4.10:80 -i $INET
iptables -A FORWARD -d 10.10.4.10 -p tcp --dport 80 -j ACCEPT
iptables -A FORWARD -s 10.10.4.10 -p tcp --sport 80 -o $INET -j ACCEPT
\end{Verbatim}

iptables -L -nv
\begin{Verbatim}
Chain INPUT (policy ACCEPT 1708 packets, 134K bytes)
 pkts bytes target     prot opt in     out     source               destination         

Chain FORWARD (policy DROP 1 packets, 60 bytes)
 pkts bytes target     prot opt in     out     source               destination         
    0     0 ACCEPT     icmp --  *      *       0.0.0.0/0            0.0.0.0/0           
    0     0 ACCEPT     all  --  tun0   *       0.0.0.0/0            0.0.0.0/0           
    0     0 ACCEPT     all  --  *      tun0    0.0.0.0/0            0.0.0.0/0           
    6   385 ACCEPT     all  --  eth1   *       0.0.0.0/0            0.0.0.0/0           state ESTABLISHED 
    0     0 ACCEPT     tcp  --  *      *       10.10.4.10           0.0.0.0/0           tcp dpt:80 
    6   337 ACCEPT     all  --  *      *       10.10.4.20           0.0.0.0/0           
    0     0 ACCEPT     tcp  --  *      *       0.0.0.0/0            10.10.4.10          tcp dpt:80 
    0     0 ACCEPT     tcp  --  *      eth1    10.10.4.10           0.0.0.0/0           tcp spt:80 

Chain OUTPUT (policy ACCEPT 1303 packets, 105K bytes)
 pkts bytes target     prot opt in     out     source               destination
\end{Verbatim}

iptables -L -nv -t nat
\begin{Verbatim}
Chain PREROUTING (policy ACCEPT 450 packets, 54516 bytes)
 pkts bytes target     prot opt in     out     source               destination         
    0     0 DNAT       tcp  --  eth1   *       0.0.0.0/0            0.0.0.0/0           tcp dpt:80 to:10.10.4.10:80 

Chain POSTROUTING (policy ACCEPT 16 packets, 828 bytes)
 pkts bytes target     prot opt in     out     source               destination         
    1    60 MASQUERADE  all  --  *      eth1    0.0.0.0/0            0.0.0.0/0           

Chain OUTPUT (policy ACCEPT 72 packets, 4607 bytes)
 pkts bytes target     prot opt in     out     source               destination
\end{Verbatim}

\section{Проверка трансляции SNAT}

Пинг yandex.ru с S11 ip - 10.10.4.10

Дамп на R1
\begin{Verbatim}
13:45:51.465234  In 10:10:10:10:20:aa ethertype IPv4 (0x0800), length 100: 
10.10.4.10 > 77.88.55.80: ICMP echo request, id 24834, seq 1, length 64
13:45:51.465280 Out fa:de:dc:30:96:57 ethertype IPv4 (0x0800), length 100: 
172.16.1.3 > 77.88.55.80: ICMP echo request, id 24834, seq 1, length 64
13:45:51.625769  In a2:91:8c:7e:80:87 ethertype IPv4 (0x0800), length 100: 
77.88.55.80 > 172.16.1.3: ICMP echo reply, id 24834, seq 1, length 64
13:45:51.625799 Out 0e:ab:f8:0c:10:4b ethertype IPv4 (0x0800), length 100: 
77.88.55.80 > 10.10.4.10: ICMP echo reply, id 24834, seq 1, length 64
\end{Verbatim}

Дамп на локальном компьютере
\begin{Verbatim}
16:47:49.981967  In fa:de:dc:30:96:57 ethertype IPv4 (0x0800), length 100: 
172.16.1.3 > 77.88.55.60: ICMP echo request, id 25346, seq 1, length 64
16:47:49.982024 Out c0:b6:f9:e9:bd:ad ethertype IPv4 (0x0800), length 100: 
192.168.184.194 > 77.88.55.60: ICMP echo request, id 25346, seq 1, length 64
16:47:50.072943  In aa:07:27:7b:fd:35 ethertype IPv4 (0x0800), length 100: 
77.88.55.60 > 192.168.184.194: ICMP echo reply, id 25346, seq 1, length 64
16:47:50.072994 Out a2:91:8c:7e:80:87 ethertype IPv4 (0x0800), length 100: 
77.88.55.60 > 172.16.1.3: ICMP echo reply, id 25346, seq 1, length 64

\end{Verbatim}


\section{Проверка правил фильтрации}
Проверка доступа к 80 порту из машины S11

telnet google.com 80
\begin{Verbatim}
Trying 216.58.209.206...
Connected to google.com.
Escape character is '^]'.
hey
HTTP/1.0 400 Bad Request
Content-Type: text/html; charset=UTF-8
Referrer-Policy: no-referrer
Content-Length: 1555
Date: Sun, 26 Dec 2021 14:08:41 GMT

<!DOCTYPE html>
<html lang=en>
  <meta charset=utf-8>
  <meta name=viewport content="initial-scale=1, minimum-scale=1, width=device-width">
  <title>Error 400 (Bad Request)!!1</title>
  <style>
    *{margin:0;padding:0}html,code{font:15px/22px arial,sans-serif}html{background:#fff;color:#222;padding:15px}body{margin:7% auto 0;max-width:390px;min-height:180px;padding:30px 0 15px}* > body{background:url(//www.google.com/images/errors/robot.png) 100% 5px no-repeat;padding-right:205px}p{margin:11px 0 22px;overflow:hidden}ins{color:#777;text-decoration:none}a img{border:0}@media screen and (max-width:772px){body{background:none;margin-top:0;max-width:none;padding-right:0}}#logo{background:url(//www.google.com/images/branding/googlelogo/1x/googlelogo_color_150x54dp.png) no-repeat;margin-left:-5px}@media only screen and (min-resolution:192dpi){#logo{background:url(//www.google.com/images/branding/googlelogo/2x/googlelogo_color_150x54dp.png) no-repeat 0% 0%/100% 100%;-moz-border-image:url(//www.google.com/images/branding/googlelogo/2x/googlelogo_color_150x54dp.png) 0}}@media only screen and (-webkit-min-device-pixel-ratio:2){#logo{background:url(//www.google.com/images/branding/googlelogo/2x/googlelogo_color_150x54dp.png) no-repeat;-webkit-background-size:100% 100%}}#logo{display:inline-block;height:54px;width:150px}
  </style>
  <a href=//www.google.com/><span id=logo aria-label=Google></span></a>
  <p><b>400.</b> <ins>That’s an error.</ins>
  <p>Your client has issued a malformed or illegal request.  <ins>That’s all we know.</ins>
Connection closed by foreign host.
\end{Verbatim}


Проверка доступа ко всем адресам из машины S12

telnet bmstu.ru 22
\begin{Verbatim}
Trying 195.19.50.250...
Connected to bmstu.ru.
Escape character is '^]'.
SSH-2.0-OpenSSH_8.0
dfd
Invalid SSH identification string.
Connection closed by foreign host.
s12:~# telnet bmstu.ru 22
Trying 195.19.50.250...
Connected to bmstu.ru.
Escape character is '^]'.
sds
SSH-2.0-OpenSSH_8.0
Invalid SSH identification string.
Connection closed by foreign host.
\end{Verbatim}



\section{Проверка доступа к внутреннему серверу}

telnet 172.16.1.3  80

\begin{Verbatim}
Trying 172.16.1.3...
Connected to 172.16.1.3.
Escape character is '^]'.
hey
<!DOCTYPE HTML PUBLIC "-//IETF//DTD HTML 2.0//EN">
<html><head>
<title>501 Method Not Implemented</title>
</head><body>
<h1>Method Not Implemented</h1>
<p>hey to /index.html not supported.<br />
</p>
<hr>
<address>Apache/2.2.9 (Debian) Server at 127.0.0.1 Port 80</address>
</body></html>
\end{Verbatim}

\end{document}
